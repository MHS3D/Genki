% !TeX root = RDT.tex
\documentclass[a4paper, 11pt]{article}    

\usepackage[ngerman]{babel}                   
\usepackage[onehalfspacing]{setspace}
\usepackage[top=2.5cm,bottom=2.5cm,left=2cm,right=2cm,marginparwidth=1.75cm]{geometry}

% Load useful packages                        
\usepackage{amsmath}                            
\usepackage{xcolor}                             
\definecolor{custom-blue}{RGB}{0,99,166} 
\usepackage{hyperref}
\hypersetup{colorlinks=true, allcolors=custom-blue}
\usepackage[default]{sourcesanspro}             
\usepackage[T1]{fontenc}                        
\usepackage{wrapfig}
\usepackage{fancyhdr}
\usepackage{longtable}
\usepackage{lastpage}
\usepackage{float}
\usepackage{tabularx} 
\usepackage[normalem]{ulem}
\useunder{\uline}{\ul}{}
\usepackage{booktabs}
\usepackage{graphicx}
\usepackage{color}
\usepackage{tabularray}
\usepackage{enumitem}
\usepackage{subcaption}


\pagestyle{myheadings}
\pagestyle{fancy}     

\definecolor{Alto}{rgb}{0.878,0.878,0.878}
\definecolor{Nobel}{rgb}{0.701,0.701,0.701}

\setlength{\headheight}{30pt}
\renewcommand{\headrulewidth}{0.5pt}
\renewcommand{\footrulewidth}{0.5pt}

\fancyhead[C]{}                                 
\fancyhead[R]{}                      
\fancyfoot[L]{}
\fancyfoot[C]{}                 
\fancyfoot[R]{\thepage/\pageref{LastPage}}

\title{
    MHS3D
}

\begin{document}

\maketitle           
\clearpage

\tableofcontents
\clearpage

%
\section{Requirements}
\begin{enumerate}
    \item Die Smartwatch soll den angebauten Pulssensor und den MPU6050 Beschleunigungssensor + Gyroskop ansteuern und deren Messdaten abfragen können.
    \item Die von der Uhr gesammelten Messdaten sollen im laufenden Betrieb an einen PC übertragen werden können. 
    \item Bewegungsrichtungen sollen live in 3D (als Linie im “Raum”) angezeigt werden.
    \item Der Puls der letzten 2 Stunden soll als Pulshöhe/Zeit als Liniendiagramm, gefärbt nach Bereichen, altersabhängig (grün bis rot) angezeigt werden.
    \item Pulsänderungen sollen als farbliche Veränderung des Bewegungsstriches im Raum visualisiert werden.
\end{enumerate}

Erweiterungsideen:
\begin{itemize}
    \item Aufnahme der Daten in Sessions mit Button an der Uhr als Start/Stop
    \item Interaktive Drehung der Ansicht möglich
    \item Mapping der Bewegungen in die Raumpläne der HAW
\end{itemize}
\clearpage


\section{Projektplan}
\begin{center}
\begin{tabularx}{\textwidth}{ |c|X|X|X| }
\hline
\textbf{KW} & \textbf{Micro} & \textbf{Magic} & \textbf{Kino} \\
\hline
\textbf{44}  & Projekt aufsetzen  &  &  \\
\hline
45  & Daten per MQTT senden, Format der Daten festlegen   & Recherche Algorithmen & Recherche: Darstellung Was und Wie?  \\
\hline
\textbf{46}  &   &  & Mockups erstellen  \\
\hline
47  & Herzschrittzähler zum Laufen kriegen   &  & Aufsetzen des Frontends  \\
\hline
\textbf{48}  & Done?  &  &  \\
\hline
49  &   &  & Testen der Darstellung mit Dummy-Daten \\
\hline
\textbf{50}  &   &  &  \\
\hline
51  &   &  & Daten von Magic empfangen und darstellen  \\
\hline
\textbf{54}  &   &  &  \\
\hline
55  &   &  & Testen  \\
\hline
\textbf{56}  &   &  &  \\
\hline
57  &   &  &  \\
\hline
\end{tabularx}
\end{center}
\clearpage

\section{Schaltplan}
\begin{figure}[H]
    \centering
    \begin{subfigure}{0.49\textwidth}
        \centering
        \includegraphics[width=\textwidth]{images/verkabelte_Smartwatch.JPEG}
        \caption{Schaltplan Smartwatch}
        \label{fig:Schaltplan Smartwatch}
    \end{subfigure}
    \begin{subfigure}{0.49\textwidth}
        \centering
        \includegraphics[width=\textwidth]{images/verkabelte_Smartwatch.JPEG}
        \caption{verkabelte Smartwatch}
        \label{fig:verkabelte Smartwatch}
    \end{subfigure}
\end{figure}

\clearpage

\section{Datenübertragung}
Die Datenübertragung findet als HTTP GET Anfrage an den Webserver der Smartwatch statt. Die Smartwatch liefert dann einen JSON String von Datensätzen.

\noindent
Ein Datensatz enthält einen Timestamp, Acceleration x,y und z Achse, Gyroskop x,y und z Achse sowie ein Pulssensor Messwert in Folgendem Format:

\clearpage

\end{document}